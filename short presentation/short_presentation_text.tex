\documentclass[11pt]{article}
\usepackage[utf8]{inputenc}
\usepackage[T1]{fontenc}
\usepackage[slovene]{babel}
\usepackage{lmodern}

\usepackage{amsmath,amssymb}
\usepackage[shortlabels]{enumitem}

\title{Quasi-random number generator}
\author{Ines Meršak}
\date{21.~4.~2017}


\begin{document}
    \maketitle

\section{Pseudo-random vs. quasi-random}
    Pseudo-random number:
    \begin{itemize}
        \item a computer-generated number
        \item appears to be random but is generated by an entirely deterministic process
        \item a pseudo-random process is easier to produce than a genuinely random one
        \item the benefit of a pseudo-random process is that it can be used again and again to produce exactly the same numbers, which is useful for testing and fixing software
    \end{itemize}

    Quasi-random number:
    \begin{itemize}
        \item also called low-discrepancy numbers
        \item the discrepancy is a measure of how inhomogeneously a set of $d$-dimensional vectors are distributed in the unit hypercube
        \item low-discrepancy means the points are distributed more uniformly, with less clusters and gaps that are typical for pseudo-random numbers
        \item unlike pseudo-random numbers, low-discrepancy numbers aim not to be serially uncorrelated but instead to take the previous draws into account when determining the next number in the sequence
    \end{itemize}
If we take a uniform random generator on $[0,1)$ and halve the interval, for each trial there is a probability of $\frac{1}{2}$ that the generated point will be in the left interval and a probability of $\frac{1}{2}$ that the point will be in the right interval. generating a point on each of these subintervals. 
Therefore, it is possible for first $n$ generated points to coincidentally all lie in the first half of the interval, while the next point still falls within the other of the two halves with probability $\frac{1}{2}$. This is not the case with the quasirandom sequences because of the low-discrepancy requirement that has an effect of points being generated in a highly correlated manner (i.e., the next point ``knows'' where the previous points are).

\section{Quasi-random numbers -- usage}
Quasi-random numbers are useful in computational problems and are especially popular for financial Monte Carlo calculations. Quasi-Monte Carlo calculations (using sequences of quasi-random numbers to compute the integral) asimptotically converge faster than normal Monte Carlo calculations using pseudo-random numbers, even for large dimensionality of drawn vectors $d$.

\section{Sobol' numbers}
\begin{itemize}
    \item we need a new unique generating integer $\gamma(n)$ for each new draw \item easy choice is $\gamma (n) = n$, another possibility is the Gray code $\gamma (n) = G(n)$, which I will not be discussing
    \item the generation is carried out on a set of integers in the interval $[1, 2^b-1]$
    \item $b$ represents the number of bits in an unsigned integer on the given computer and is typically 32
    \item denote $x_{nk}$ as the $n$th draw of Sobol' integer in dimension $k$
    \item a set of $b$ \emph{direction integers} for each dimension $k$, which are the basis of the number generation
    \item there are some additional constraints on the direction integers which I will not discuss
    \item for each dimension, we select a primitive polynomial modulo two and calculate the direction integers using the coefficients of the polynomial and binary addition
    \item from there, we calculate $x_{nk}$: depending on which bits in the binary representation of $\gamma(n)$ are set, the direction integers are XORed to produce the Sobol' integer $x_{nk}$
\end{itemize}

\section{Project timeline}
    Work done so far:
    \begin{itemize}
        \item reading the source material
        \item getting familiar with C++
    \end{itemize}
    
    Plan for the rest of the project: 
    \begin{itemize}
        \item implement Sobol' number generator with Gray code
        \item test the generator with quasi-Monte Carlo integration
        \item compare results with the parallel version
    \end{itemize}
\end{document}